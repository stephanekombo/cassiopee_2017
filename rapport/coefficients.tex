\section{Décomposition de l'onde plane sur le frame de fenêtre gaussiennes}\label{decomposition}


\subsection{Simplifications préalables}


Considérons le champ $\v{E} = \v{E}_0\exp\lb i(xk_{ix}+yk_{iy}+zk_{iz})\rb$ défini en \eqref{OPPH}.
Pour sa composante vectorielle, on peut écrire $\v{E}_0=E_{0x}\h{x}+E_{0y}\h{y}$,
avec $E_{0x}$ et $E_{0y}$ constants. Par linéarité des opérateurs que nous allons utiliser,
il suffit de décomposer $f(x,y,z) = \exp\lb i(xk_{ix}+yk_{iy}+zk_{iz})\rb$ (terme ondulatoire)
pour obtenir la décomposition du champ global.

Il suffit de le faire sur le plan d'équation $z=0$;
en effet, par définition de l'onde plane, la connaissance de celle-ci sur un front d'onde permet d'en déduire
le champs en tout point $M(x,y,z)$ de l'espace par translation.


\subparagraph{}

Considérons donc  $f(x,y) = f(x,y,0) = \exp\lb i(xk_{ix}+yk_{iy})\rb$.


\subsection{Décomposition du spectre du terme ondulatoire}\label{decompositionSpectre}


Pour le spectre $\t{f}$ de cette fonction, la théorie des frames de Gabor nous garantit l'existence
d'une décomposition sur le frame $(\t{w}_{nmqp}) = (\t{w}^x_{nm}\t{w}^y_{qp})$ de la forme:

\be
\t{f}=\Sum\t{A}_{mnpq}\t{w}^x_{nm}\t{w}^y_{qp},\text{ où }
\t{A}_{mnpq}=<\t{f},\h{\t{w}_{nm}^x\t{w}_{qp}^y}>=<\t{f},\h{\t{w}_{nm}^x}\h{\t{w}_{qp}^y}>.
\label{a}
\tag{a}
\ee

Il s'agit d'une récriture de \eqref{analyse} et \eqref{synthese} vue depuis le domaine spectral. 
On a donc:

\begin{align}
  \t{A}_{mnpq}  &=\Int\t{f}\lc \h{\t{w}_{nm}^x}\h{\t{w}_{qp}^y}\rc^\ast&
  \text{par définition de }<.,.>\nonumber\\
      &=\Int\Int\t{f}(k_x,k_y)\lc\h{\t{w}}_{nm}(k_x)\h{\t{w}}_{qp}(k_y)\rc^\ast dk_xdk_y &
      \text{ }\nonumber\\
  \t{A}_{mnpq}  &=\Int\Int\t{f}(k_x,k_y)\h{\t{w}}_{nm}(k_x)^\ast\h{\t{w}}_{qp}(k_y)^\ast dk_xdk_y &
  \text{ }\label{formuletA}
\end{align}


On cherche donc à calculer les coefficients $(\t{A}_{mnpq})_{m,n,p,q\in\Z}$ en utilisant \eqref{formuletA}.


\subsubsection{Calcul préliminaire}


\begin{align}
  \t{f}(k_x,k_y)  &= \Int\Int f(x,y)\e^{-i(xk_x+yk_y)}dxdy &
  \text{par \eqref{tf}}\nonumber\\
      &= \Int\Int\e^{i(xk_{ix}+yk_{iy})}\e^{-i(xk_x+yk_y)}dxdy &
      \text{par réécriture de \eqref{OPPH}}\nonumber\\
      &= \Int\Int\e^{-ix(k_x-k_{ix})}\e^{-iy(k_y-k_{iy})}dxdy & 
      \text{ }\nonumber\\
      &= \Int 1\cdot\e^{-ix(k_x-k_{ix})}dx\Int 1\cdot\e^{-iy(k_y-k_{iy})}dy &
      \text{par théorème de Fubini}\nonumber\\
      &= \F\lb \1\rb(k_x-k_{ix})\F\lb \1\rb(k_y-k_{iy}) &
      \text{où $\1$ désigne la fonction constante égale à $1$}\nonumber\\
  \t{f}(k_x,k_y)  &= \delta (k_x-k_{ix})\delta (k_y-k_{iy}) &
  \text{par  \eqref{t1}}\label{spectre}
\end{align}

Cette expression va considérablement simplifier l'expression de $\t{A}_{mnpq}$, en réduisant l'intégrale double
à la valeur de l'intégrande évaluée en un point du domaine spectral (propriété de l'impulsion de Dirac).


\subsubsection{Retour à la décomposition du terme ondulatoire}


\begin{align*}
  \t{A}_{mnpq} &= \Int\Int\delta (k_x-k_{ix})\delta (k_y-k_{iy})
  \h{\t{w}}_{nm}(k_x)^\ast\h{\t{w}}_{qp}(k_y)^\ast dk_xdk_y &
  \text{par \eqref{formuletA} et \eqref{spectre}}\\
   &= \Int\delta (k_x-k_{ix})\h{\t{w}}_{nm}(k_x)^\ast dk_x
  \Int\delta (k_y-k_{iy})\h{\t{w}}_{qp}(k_y)^\ast dk_y &
  \text{par théorème de Fubini}
\end{align*}

Or par définition, $\delta$ est telle que : $\Int\delta (k_x-k_{ix})\h{\t{w}}_{nm}(k_x)^\ast dk_x =
\h{\t{w}}_{nm}(k_{ix})^\ast $

Et de même pour la variable $y$: $\Int\delta (k_y-k_{iy})\h{\t{w}}_{qp}(k_y)^\ast dk_y =
\h{\t{w}}_{qp}(k_{iy})^\ast$

On obtient donc simplement l'expression:

\be
\t{A}_{mnpq} = \h{\t{w}}_{nm}^\ast(k_{ix})\h{\t{w}}_{qp}^\ast(k_{iy}) = 
\h{\t{w}}(k_{ix}-n\b{k}_x)^\ast\h{\t{w}}(k_{iy}-q\b{k}_y)^\ast\e^{+i\lc m\b{x}k_{ix}+p\b{y}k_{iy}\rc}
\label{tA}
\ee


\subsection{Décomposition effective de l'onde}\label{decompositionOnde}


On a donc, à l'issu de \ref{decompositionSpectre}, la décomposition de $\t{f}$ suivante:

\be
\t{f}(k_x,k_y) = \Sum \t{A}_{mnpq}\t{w}^x_{nm}(k_x)\t{w}^y_{qp}(k_y),
\text{ où }\t{A}_{mnpq} = \h{\t{w}}_{nm}^\ast(k_{ix})\h{\t{w}}_{qp}^\ast(k_{iy})
\ee

Cette décomposition a son analogue dans le domaine spatial
\footnote{\eqref{b} peut être vu comme la transformée de Fourier inverse de \eqref{a}, réécrite grace à \eqref{wmnt}}
(voir \eqref{analyse} et \eqref{synthese}):

\be
f(x,y) = \Sum A_{mnpq}w^x_{mn}(x)w^y_{pq}(y),
\text{ où }A_{mnpq} = \t{A}_{mnpq}\e^{-i(mn\b{x}\b{k}_x +pq\b{y}\b{k}_y)}
\label{b}
\tag{b}
\ee


La reconstruction de l'onde $\v{E}$ se fait donc finalement avec les formules:

\begin{empheq}[box=\fbox]{equation}
\v{E}(x,y,z=0)=\lc\Sum A_{mnpq}w^x_{mn}(x)w^y_{pq}(y)\rc\v{E}_0
\label{vE}
\end{empheq}

où: 
 
\begin{empheq}[box=\fbox]{equation}
A_{mnpq}=\h{\t{w}}_{nm}^\ast(k_{ix})\h{\t{w}}_{qp}^\ast(k_{iy})\e^{-i(mn\b{x}\b{k}_x +pq\b{y}\b{k}_y)}
\label{A}
\end{empheq}


\subparagraph{}

Dans le cadre d'un calcul formel, il est difficile de faire plus explicite et concis
car le dual d'une fonction quelconque ne dispose pas d'expression analytique connue.
Ce sont donc ces formules (encadrées) que nous utiliserons en \ref{implementation} lors de la validation numérique:
l'algorithme appliqué va nous permettre d'approximer numériquement $\h{\t{w}}$ (consulter \cite{TheseLugara},
\cite{FiniteFrameTheory} et \cite{Algo} pour des démonstrations et détails sur cet algorithme).


\newpage