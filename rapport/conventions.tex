\section*{Conventions et notations}


  Dans un contexte de physique, nous utiliserons presque exclusivement la notation impropre $w(x)$
  pour désigner la fonction $x\rightarrow w(x)$, afin de mettre en évidence la variable $x$ considérée.
  
  Sauf mention contraire, toute relation dépendant d'une variable $x$
  sera valable pour toute autre variable.  
  Selon le cadre (monovarié ou pas, ou en cas de notation supplémentaire), on désignera indifféremment
  $w$, $w_x$, $w^x$, $w(x)$, $w^x(x)$, ou $w_x(x)$, les indices supplémentaires étant utilisés
  surtout dans un contexte multivarié.

  Afin de simplifier les notations, toute intégrale sans borne sera considérée sur $\R$ entier:
  $\Int=\Int_{-\infty}^{\infty}$
  
  De même, sans précision supplémentaire, toute somme est considérée sur l'ensemble d'indices de sommation
  le plus grand possible:
  pour $a,\dots,z$ entiers, $\Sum=\Sum_{a,\dots,z}=\Sum_{a=-\infty}^{\infty}\dots\Sum_{z=-\infty}^{\infty}$

  Plus conventionnellement, nous adopterons les notations définies sur cette page.
  Pour celles présentées ensuite dans la suite du document, nous tenterons de rester en cohérence
  avec les thèses de Delphine Lugara et Ihssan Ghannoum (respectivement \cite{TheseLugara} et \cite{TheseGhannoum}),
  car le travail présenté ici se place dans la continuité de celles-ci.


\begin{center}
$
\ba{lcllc}
  i &\ &\  &
  \text{Nombre imaginaire pur tel que } i^2 = -1&\\
  h^\ast, f^\ast &\ &\ &
  \text{Complexe, fonction conjugué.e} &\\
  \e^{-i\omega t} &\ &\  &
  \text{Convention en régime harmonique} &\\
  \v{a}\cdot\v{b} &\ &\  &
  \text{Produit scalaire de deux vecteurs} &\\
  <f,g> = \Int fg^\ast = \Int f(x)g(x)^\ast dx &\ &\ &
  \text{Produit scalaire de deux fonctions monovariées} &\\
  <f,g> = \Int\Int f(x,y)g(x,y)^\ast dxdy &\ &\ &
  \text{Produit scalaire de deux fonctions bivariées} &\\
  \t{f}(k_x) = \F{\lb f(x)\rb}(k_x) = \Int f(x)\e^{-ixk_x}dx &\ &\ &
  \text{Transformée de Fourier de $f$}&\\
  \t{f}(k_x,k_y) = \Int\Int f(x,y)\e^{-i(xk_x+yk_y)}dxdy &\ &\ &
  \text{Transformée de Fourier $2D$}&\\
  f(x) = \F^{-1}{\lb\t{f}(k_x)\rb}(x) = \ff[1]{2\pi}\Int\t{f}(k_x)\e^{ixk_x}dx &\ &\ &
  \text{Transformée de Fourier inverse de $f$}&\\
  f(x,y) = \ff[1]{(2\pi)^2}\Int\Int\t{f}(k_x,k_y)\e^{i(xk_x+yk_y)}dx &\ &\ &
  \text{Transformée inverse $2D$}&\\
  \llb m,n\rrb &\ &\  &
  \text{Intervalle d'entiers }&\\
  \lfloor x\rfloor &\ &\  &
  \text{Partie entière par défaut}&	\\
  \lceil x\rceil &\ &\  &
  \text{Partie entière par excès}&  
\ea
$
\end{center}


Notons que cette convention, pour la transformée de Fourier et son inverse, a un impact sur certaines
propriétés de la transformée. En cas de doute, on pourra consulter le formulaire en annexe \ref{formulaire}
en plus de cette section de conventions.


\newpage
