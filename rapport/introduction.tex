\section{Introduction}\label{introduction}


\subparagraph{}

  Ce rapport est une tentative de contribution aux travaux de Christine Letrou sur les faisceaux gaussiens.
La méthode de Lancer de Faisceaux Gaussiens (LFG) permet de calculer des champs électromagnétiques se propageant
dans des environnements complexes comportant des obstacles.
La question principale qui a mené à la proposition de ce travail porte sur l'intérêt de cette méthode pour les calculs
de Surface Équivalente Radar (SER) monostatique en environnement urbain.
Le but est donc ici la détermination du nombre de faisceaux à lancer suivant le type d’environnement
(densité et disposition des bâtiments) et selon le champ électromagnétique considéré, ici
une onde électromagnétique plane progressive harmonique (OPPH).


\subparagraph{}

La démarche justifiant l'approche utilisée dans ce travail est classique: afin de résoudre les problèmes de propagation
d'un champ électromagnétique, il est d’usage de décomposer ce champ sur une \og base \fg\ d’objets élémentaires
dont on connaît le comportement, afin d’en déduire les propriétés globales par théorème de superposition.
Il faut donc choisir la famille d'éléments de base de façon pertinente, afin de simplifier le problème étudié.
On fait ici le choix d'un ensemble très particulier de fonctions gaussiennes,
appelé un \emph{frame de Gabor à fenêtres gaussiennes}, pour au moins ces deux raisons essentielles:

\begin{itemize}

 \item Il existe plusieurs représentations et définitions des faisceaux gaussiens; dans notre cas,
chacun des faisceaux correspondra à une partie du champ incident, fenêtré par une fonction gaussienne.
On dira alors que chaque fenêtre du frame rayonne sous la forme d'un faisceau gaussien.

\item Dans un contexte d'analyse spectrale, incontournable en electromagnétisme,
 nous allons nous interesser aux propriétes \emph{spatiales} des faisceaux étudiés, et nous allons avoir besoin de
 conserver la connaissance de certaines propriétés des faisceaux dans le domaine spectral, ou domaine transformé.
 Il se trouve que les frames de Gabor à fenêtres gaussiennes sont les objets élémentaires qui nous accordent le meilleur
 compromis entre connaissance des objets globaux à la fois dans le domaine spatial et le domaine spectral 
 (on parle de bonne localisation spatiale et spectrale).
 
\end{itemize}

Par soucis de concision, nous supposerons connue la théorie générale des frames.
Bien que \cite{SlidesLetrou} résume la plupart des résulats fondamentaux utilisés dans le présent rapport,
il peut être profitable de se référer au premier chapitre de \cite{FiniteFrameTheory} pour avoir une introduction
plus détaillée, incluant des prérequis d'algèbre linéaire.

\cite{GaborAlgebra} présente les éléments fondamentaux de la théorie des frames de manière plus théorique.
 
 

\subparagraph{}

  Pour réaliser ce projet de calcul de SER, il faut donc, pour cette OPPH représentée par sa composante électrique $\v{E}$,
  déterminer une décomposition \emph{exploitable} de la forme $\v{E} = \Sum\v{A}_{mnpq}w_{mnpq}$,
où le frame de Gabor $(w_{mnpq})_{m,n,p,q\in\Z}$ est construit à partir d'une fenêtre $w$ gaussienne,
et les $\v{A}_{mnpq}$ seront les coefficients de la décomposition, soit l'objet concret de tout le travail
présenté dans ce rapport.

La proposition suivante, démontrée pas à pas dans \cite{TheseLugara}, justifie la validité de notre démarche:

\begin{prop}

  Les coefficients $\v{A}_{mnpq}$ caractérisent complétement $\v{E}$, avec une reconstruction numériquement stable
  de $\v{E}$ à partir de ces coefficients, \emph{si et seulement si} $(w_{mnpq})$ forme un frame de Gabor.  

  De même pour le problème dual, tout champ $\v{E}$ est exprimable en tant que superposition de fenêtres $w_{mnpq}$
 \emph{si et seulement si} $(w_{mnpq})$ forme un frame de Gabor.

 \end{prop}


\subparagraph{}

Dans un premier temps, la suite du document propose donc une expression la plus explicite possible des $\v{A}_{mnpq}$.
Leur expression faisant intervenir la fonction duale $\h{\t{w}}$, pour laquelle on ne dispose pas
d'expression analytique, nous seront amenés à l'approximer grâce à un algorithme itératif décrit dans \cite{TheseLugara}.
Le second objectif de ce rapport sera donc la validation numérique de cette approximation, c'est-à-dire la vérification
de la bonne reconstruction de notre OPPH avec une précision suffisante par utilisation de cette démarche.


\newpage